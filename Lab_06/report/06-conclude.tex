\addchap{Заключение}

В ходе выполнения данной лабораторной работы были выполнены следующие задачи:
\begin{itemize}
	\item изучено понятие муравьиного алгоритма на примере решения задачи коммивояжёра;
	\item изучено решение этой задачи с помощью метода полного перебора;
	\item составлены схемы данных алгоритмов;
	\item реализованны разработанные версии алгоритмов;
	\item проведена параметризацию муравьиного алгоритма для выбранных классов задач;
	\item проведен сравнительный анализ скорости работы реализованных алгоритмов;
	\item описаны и обоснованы полученные результаты.
\end{itemize}

Экспериментально были установлены различия в производительности муравьиного алгоритма и метода полного перебора. Для графов с небольшим количеством вершин выгоднее использовать второй алгоритм, однако с увеличением числа вершин в графе, время работы реализации полного перебора значительно возрастает, поэтому для графов, в которых количество вершин > 10, выгоднее использовать муравьиный алгоритм. Уже при n = 11 данный алгоритм оказывается эффективнее в $\approx$ 12 раз. Выбор параметров для муравьиного
алгоритма зависит от значений матрицы смежности и от конкретного условия задачи.