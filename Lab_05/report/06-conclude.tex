\addchap{Заключение}

В ходе выполнения данной лабораторной работы были выполнены следующие задачи:
\begin{itemize}
	\item изучены основы конвейерной обработки данных;
	\item составлены схемы алгоритмов основной работы конвейера и его этапов;
	\item реализованны разработанные алгоритмов;
	\item проведен сравнительный анализ конвейерной и последовательной реализаций по затрачиваемым ресурсам (время и память);
	\item описаны и обоснованы полученные результаты.
\end{itemize}

В результате исследований можно прийти к выводу, что использование конвейерной обработки данных оптимальнее последовательного в тех случаях, когда выполняемые задачи намного больше по времени, чем время затрачиваемое на реализацию конвейера (работа с потоками, администрирование очередей по лентам и т.д.). По памяти же конвейерная реализация оказывается более затратной, так как выделяется память под дополнительные структуры данных, используемых для организации работы системы.