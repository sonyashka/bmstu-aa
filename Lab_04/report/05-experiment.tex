\chapter{Исследовательский раздел}

В данном разделе будет проведен сравнительный анализ алгоритмов по
времени и затрачиваемой памяти.

\section{Технические характеристики}

Тестирование выполнялось на устройстве со следующими характеристиками: 

\begin{itemize}
	\item Операционная система Windows 10 \cite{windows10}
	\item Память 8 Гб
	\item Процессор Intel Core i3 7020U, 2.3 ГГц \cite{intel}
\end{itemize}

Во время проведения эксперимента устройство не было нагружено сторонними задачами, а также было подключено к блоку питания.

Замеры времени осуществлялись с помощью утилиты библиотеки windows.h. Результат был умножен на $1e6$, чтобы были получены микросекунды.

\captionsetup{singlelinecheck = false, justification=raggedright}
\begin{lstlisting}[label=time, caption=Замер времени в микросекундах]
{
	LARGE_INTEGER frequency;
    LARGE_INTEGER start;
    LARGE_INTEGER end;
    double interval;
    QueryPerformanceFrequency(&frequency);
    QueryPerformanceCounter(&start);
    res = refineIntegral(a, b, func1);
    QueryPerformanceCounter(&end);
    interval = (double) (end.QuadPart - start.QuadPart)\
            / frequency.QuadPart * MCS;
}
\end{lstlisting}
\captionsetup{singlelinecheck = false, justification=centering}

При проведении эксперимента была отключена оптимизация командой компилятору -o0 \cite{std=c99}.

\section{Оценка времени работы алгоритмов}

В таблице 4.1 представлены замеры времени работы алгоритмов на интервалах интегрирования длинной от 10 до 100. Каждое значение было получено усреднением по 100 замерам. Максимальное количество запускаемых потоков было выбрано 16, так как процессор содержит 4 логических ядра \cite{intel}.

\begin{table}[H]
	\centering
	\caption{Временные замеры работы алгоритмов}
	\begin{tabular}{l|r|r|r|r|r|r}
		\text{L} & \text{Последоват.} & \text{1 п.} & \text{2 п.} & \text{4 п.}  & \text{8 п.} & \text{16 п.}\\
		\hline
		10 & 831.67 & 802.52 & 315.80 & 95.74 & 93.05 & 127.32\\
20 & 6419.24 & 6195.73 & 2307.03 & 681.79 & 559.40 & 675.64\\
30 & 26243.15 & 24761.44 & 9448.83 & 2136.75 & 1662.55 & 1955.23\\
40 & 53126.95 & 49403.12 & 18451.95 & 5417.52 & 4310.91 & 5041.34\\
50 & 111396.90 & 110161.73 & 36901.99 & 10837.60 & 8203.71 & 9651.27\\
60 & 115938.46 & 104935.93 & 75092.23 & 17391.07 & 13456.72 & 15761.32\\
70 & 114874.66 & 99674.98 & 123571.19 & 30062.82 & 23765.29 & 28051.00\\
80 & 126957.68 & 99492.35 & 149183.09 & 44076.26 & 36326.88 & 43359.88\\
90 & 142689.09 & 98890.41 & 149129.21 & 72630.98 & 60292.21 & 70491.01\\
100 & 168881.42 & 99556.42 & 200810.29 & 91269.81 & 78802.60 & 93109.37\\
	\end{tabular}
\end{table}

\section{Вывод}

По результатам исследования можно сделать вывод, что самым быстрым для процессора является использование 8 потоков. Если количество потоков увеличить до 16 время увеличивается, так как за квант процессорного времени процессы не успевают обработаться, из-за чего увеличивается объем "распаковок" и "запаковок" используемых процессором данных. 