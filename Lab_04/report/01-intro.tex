\setcounter{page}{4}
\addchap{Введение}

В настоящее время компьютерные системы оперируют большими объемами данных. Над этими данными проводится большой объем различного рода вычислений. Для того, чтобы они выполнянлись быстрее, было придумано параллельное программирование.

Его суть заключается в том, чтобы относительно равномерно разделять нагрузку между потоками ядра. Каждое из ядер процессора может обрабатывать по одному потоку, поэтому когда количество потоков на ядро становится больше, происходит квантование времени. Это означает, что на каждый процесс выделяется фиксированная величина времени (квант), после чего в течение кванта обрабатывается следующий процесс. Таким образом создается видимость параллельности. Тем не менее, данная оптимизация может сильно ускорить вычисления. 

Целью данной лабораторной работы является получение навыков параллельного программирования. Для достижения поставленной цели необходимо выполнить следующие задачи:
\begin{itemize}
	\item изучить последовательный и параллельный варианты выбранного алгоритма;
	\item составить схемы данных алгоритмов;
	\item реализовать разработанные версии алгоритма;
	\item провести сравнительный анализ реализаций по затрачиваемым ресурсам (время и память);
	\item описать и обосновать полученные результаты.
\end{itemize}