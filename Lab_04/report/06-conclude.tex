\addchap{Заключение}

В ходе выполнения данной лабораторной работы были выполнены следующие задачи:
\begin{itemize}
	\item изучены последовательный и параллельный варианты алгоритма вычисления определенного интеграла;
	\item составлены схемы данных алгоритмов;
	\item реализованны разработанные версии алгоритмов;
	\item проведен сравнительный анализ алгоритмов по затрачиваемым ресурсам (время и память);
	\item описаны и обоснованы полученные результаты.
\end{itemize}

В результате исследований можно прийти к выводу, что использование параллельного алгоритма вычисления определенного интеграла по времени является в 10 раз оптимальнее последовательного (при использовании 8 потоков). По памяти же многопоточная реализация оказывается более затратной, так как выделяется память под массив указателей на потоки и массив с данными для каждого потока отдельно.