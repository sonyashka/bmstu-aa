\setcounter{page}{4}
\addchap{Введение}

В настоящее время компьютерные системы оперируют большими объемами данных. Все они могут иметь разное представление, однако порядок обработки нередко зависит от размещения данных относительно друг друга. Для оптимизации этого процесса используется предварительная сортировка.

Сортировка - алгоритм упорядочивания элементов в порядке возрастания или убывания. Она применяется в различных областях, будь то группировка данных, поиск общих элементов в нескольких последовательностях или же поиск какой-либо информации.

Алгоритмы сортировки характеризуются временем выполнения и объемом затрачиваемой памяти. Они определяются как зависимость от длины сортируемой последовательности данных. Также выделяют устойчивую и неустойчивую сортировки. Устойчивая не меняет положение данных, с одинаковым значением.

Целью данной лабораторной работы является анализ различных видов сортировки. Для достижения поставленной цели необходимо выполнить следующие задачи:
\begin{itemize}
	\item Изучить три выбранных алгоритма сортировки;
	\item Составить схемы данных алгоритмов;
	\item Реализовать разработанные алгоритмы сортировок;
	\item Провести сравнительный анализ алгоритмов по затрачиваемым ресурсам (время и память);
	\item Описать и обосновать полученные результаты.
\end{itemize}