\chapter{Исследовательский раздел}\label{sec:exp}

В данном разделе будет проведен сравнительный анализ алгоритмов по времени и затрачиваемой памяти.

\section{Технические характеристики}

Тестирование выполнялось на устройстве со следующими характеристиками: 

\begin{itemize}
	\item Операционная система Windows 10
	\item Память 8 Гб
	\item Процессор Intel Core i3 7020U, 2.3 ГГц
\end{itemize}

Во время проведения эксперимента устройство не было нагружено сторонними задачами, а также было подключено к блоку питания.

Замеры процессорного времени проводились с помощью ассемблерной вставки, вычисляющей затраченное процессорное время в тиках.

\begin{lstlisting}[label=tick, caption=Ассемблерная вставка замера процессорного времени в тиках]
unsigned long long tick(void)
{
    unsigned long long d;
    __asm__ __volatile__ ("rdtsc": "=A" (d));
    return d;
}
\end{lstlisting}

\section{Таблица времени выполнения алгоритмов}

В таблице 4.1 представлены замеры процессорного времени работы алгоритмов на словах длиной до 10 символов. Каждое значение было получено усреднением по 100 замерам.

\begin{table}[H]
	\begin{center}
		\caption{Результаты замеров времени}
		\begin{tabular}{c|c|c|c|c}
			Длина & Л.с 2 стр. & Рек.Л.без кэша & Рек.Л.с кэшем & Рек.Д.-Лев.без кэша\\
			\hline
			1 & 706 & 103 & 397 & 108\\
2 & 798 & 626 & 1186 & 653\\
3 & 1452 & 12172 & 2406 & 4179\\
4 & 1455 & 18138 & 2829 & 17291\\
5 & 2676 & 83102 & 4389 & 79969\\
6 & 3624 & 431378 & 5371 & 431347\\
7 & 5657 & 2262049 & 7843 & 2286692\\
8 & 9671 & 14404315 & 17058 & 14350247\\
9 & 12395 & 25646236 & 22531 & 25542013\\
		\end{tabular}
	\end{center}
\end{table}

\section{Графики функций}

На рисунке 4.1 представлен график сравнения времени работы рекурсивных алгоритмов поиска расстояния Левенштейна с кэшем в форме матрицы и без кэша. Как видно, наличие кэша очень сильно ускоряет вычисления, избавляя от большого количества повторных вызовов функции.

\begin{figure}[H]
	\captionsetup{singlelinecheck = false, justification=centering}
	\centering
	\begin{tikzpicture}
			\begin{axis}[
				xlabel={длина слова},
				ylabel={время, тики},
				width=0.95\textwidth,
				height=0.4\textheight,
				xmin=0, xmax=10,
				legend pos=north west,
				xmajorgrids=true,
				grid style=dashed,
				]
				\addplot[
				color=blue,
				mark=asterisk
				]
				table [x=N, y=time]{
					N time
					1 103
					2 626
					3 12172
					4 18138
					5 83102
					6 431378
					7 2262049
					8 14404315
					9 25646236
};
				\addplot[
				color=red,
				mark=asterisk
				]
				table [x=N, y=time]{
					N time
					1 397
					2 1186
					3 2406
					4 2829
					5 4389
					6 5371
					7 7843
					8 17058
					9 22531
};
				\
				\legend{Рек. Лев. без кэша, Рек. Лев. с кэшем в ф. матрицы}
			\end{axis}
		\end{tikzpicture}
		\caption{Сравнение рекурсивных алгоритмов поиска расстояния Левенштейна с кэшем и без кэша}
\end{figure}

На рисунке 4.2 представлен график сравнения времени работы рекурсивных алгоритмов поиска расстояний Левенштейна и Дамерау-Левенштейна без использования кэша. Как можно заметить, они практически идентично накладываются друг на друга.

\begin{figure}[H]
	\captionsetup{singlelinecheck = false, justification=centering}
	\centering
	\begin{tikzpicture}
			\begin{axis}[
				xlabel={длина слова},
				ylabel={время, тики},
				width=0.95\textwidth,
				height=0.4\textheight,
				xmin=0, xmax=10,
				legend pos=north west,
				xmajorgrids=true,
				grid style=dashed,
				]
				\addplot[
				color=blue,
				mark=asterisk
				]
				table [x=N, y=time]{
					N time
					1 103
					2 626
					3 12172
					4 18138
					5 83102
					6 431378
					7 2262049
					8 14404315
					9 25646236
};
				\addplot[
				color=red,
				mark=asterisk
				]
				table [x=N, y=time]{
					N time
					1 108
					2 653
					3 4179
					4 17291
					5 79969
					6 431347
					7 2286692
					8 14350247
					9 25542013
};
				\
				\legend{Рек. Лев. без кэша, Рек. Дам.-Лев. без кэша}
			\end{axis}
		\end{tikzpicture}
		\caption{Сравнение рекурсивных алгоритмов поиска расстояния Левенштейна и Дамерау-Левенштейна без кэша}
\end{figure}

\section{Использование памяти}

Так как алгоритмы поиска расстояния Левенштейна и Дамерау-Левенштейна не отличаются друг от друга с точки зрения использования памяти, рассмотрим разницу рекурсивной и нерекурсивной реализации с кэшем.

Пусть s1 и s2 строки длины i и j соответственно.

Тогда для нерекурсивного алгоритма с кэшем в форме двух строк будет затрачена память на:

\begin{itemize}
	\item Строки s1 и s2 - (i + j) * sizeof(char)
	\item Длины строк i и j - 2 * sizeof(int)
	\item Кэш в 2 строки размерностью j + 1 - 2 * (j + 1) * sizeof(int)
	\item вспомогательные переменные - 3 * sizeof(int)
\end{itemize}

Для рекурсивных реализаций глубина стека вызовов равна сумме длин строк i + j.

Для рекурсивного алгоритма Левенштейна без кэша будет затрачена память:

\begin{itemize}
	\item Строки s1 и s2 - (i + j) * sizeof(char)
	\item Длины строк i и j - 2 * sizeof(int)
	\item Вспомогательные переменные - 2 * sizeof(int)
	\item Адрес возврата
\end{itemize}

Для рекурсивного алгоритма Левенштейна с кэшем в форме матрицы будет затрачена память:

\begin{itemize}
	\item Строки s1 и s2 - (i + j) * sizeof(char)
	\item Длины строк i и j - 2 * sizeof(int)
	\item Вспомогательные переменные - 2 * sizeof(int)
	\item Память под матрицу кэша - (i + 1) * (j + 1) * sizeof(int)
	\item Указатель на матричный кэш - 4 байта
	\item Адрес возврата
\end{itemize}

Для рекурсивного алгоритма Дамерау-Левенштейна без кэша будет затрачена память:

\begin{itemize}
	\item Строки s1 и s2 - (i + j) * sizeof(char)
	\item Длины строк i и j - 2 * sizeof(int)
	\item Вспомогательные переменные - 5 * sizeof(int)
	\item Адрес возврата
\end{itemize}

\section{Вывод}

Рекурсивные реализации поиска редакционного расстояния по времени работы оказываются на порядок дольше. На словах длиной до 10 символов виден выигрыш нерекурсивной реализации с кэшем в форме двух строк.

Также нетрудно заметить, что при добавлении кэша в рекурсивный алгоритм Левенштейна, время его выполнения значительно уменьшается за счет отсутствия повторных вычислений.

Касаемо алгоритма Дамерау-Левенштейна можно сказать что по скорости он сравним с рекурсивным Левенштейна без кэша.

Однако несмотря на затраты по времени, рекурсивные алгоритмы выигрывают по затрачиваемой памяти, так как она растет как сумма длин строк, в то время как в итеративном алгоритме - произведение длин строк.