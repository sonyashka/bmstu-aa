\addchap{Заключение}

В ходе выполнения лабораторной работы были решены следующие задачи:

\begin{itemize}
	\item Изучены алгоритм поиска редакционного расстояния Левенштейна и Дамерау-Левенштейна;
	\item Получены и применены навыки динамического программирования для некоторых реализаций;
	\item Реализованы алгоритмы:
	\begin{itemize}
		\item Нерекурсивный поиска расстояния Левенштейна с кэшем в форме двух строк;
		\item Рекурсивный поиска расстояния Левенштейна без кэша;
		\item Рекурсивный поиска расстояния Левенштейна с кэшем в форме матрицы;
		\item Рекурсивный поиска расстояния Дамерау-Левенштейна без кэша.
	\end{itemize}
	\item Проведено экспериментальное подтверждение различий алгоритмов по временной характеристике и по памяти;
	\item Подготовлен отчет по лабораторной работе.
\end{itemize}

В результате исследований можно прийти к выводу, что использование рекурсивных алгоритмов требует использования кэша для сокращения времени работы, а нерекурсивный алгоритм сильно проигрывает по используемой памяти в зависимости от длин анализируемых строк.