\chapter{Технологический раздел}

В данном разделе будут приведены требования к ПО, средства его реализации и листинга кода алгоритмов, а также рассмотрены тестовые случаи.

\section{Требования к ПО}

Программное обеспечение должно удовлетворять следующим требованиям:
\begin{itemize}
	\item ПО принимает на вход текстовые данные в любой раскладке, длиной не более 100 символов;
	\item ПО возвращает редакционное расстояние.
\end{itemize}

\section{Средства реализации} 

Для реализации ПО был выбран компилируемый язык C. В качестве среды разработки - QtCreator. Оба средства были выбраны из тех соображений, что навыки работы с ними были получены в более ранних курсах, а значит это упростит разработку.

\section{Листинги кода}

В листингах 3.1 - 3.4 приведены реализации алгоритмов, изученных в аналитическом разделе.

\begin{lstlisting}[label=levCacheTwoRows, caption=Нерекурсивный алгоритм поиска расстояния Левенштейна с кэшем в форме 2 строк]
int levCacheTwoRows(char *s1, int i, char *s2, int j)
{
    int *a1 = malloc(sizeof(int) * (j + 1));
    int *a2 = malloc(sizeof(int) * (j + 1));
    if (!a1 && !a2)
        return -1;

    for (int m = 0; m < j + 1; m++)
        a1[m] = m;

    int flag;
    for (int m = 1; m < i + 1; m++)
    {
        a2[0] = a1[0] + 1;
        for (int n = 1; n < j + 1; n++)
        {
            if (s1[m - 1] == s2[n - 1])
                flag = 0;
            else
                flag = 1;
            a2[n] = min(a2[n - 1] + 1, a1[n] + 1, a1[n - 1] + flag);
        }
        for (int n = 0; n < j + 1; n++)
            a1[n] = a2[n];
    }
    int result = a2[j];
    free(a1);
    free(a2);
    return result;
}
\end{lstlisting}

\begin{lstlisting}[label=damLevRecWithoutCache, caption=Рекурсивный алгоритм поиска расстояния Левенштейна без кэша]
int levRecWithoutCache(char *s1, int i, char *s2, int j)
{
    if (i == 0)
        return j;
    else if (j == 0)
        return i;

    int flag;
    if (s1[i - 1] == s2[j - 1])
        flag = 0;
    else
        flag = 1;
    int result = min(levRecWithoutCache(s1, i - 1, s2, j) + 1,\
                     levRecWithoutCache(s1, i, s2, j - 1) + 1,\
                     levRecWithoutCache(s1, i - 1, s2, j - 1) + flag);
    return result;
}
\end{lstlisting}

\begin{lstlisting}[label=levRecWithCache, caption=Рекурсивный алгоритм поиска расстояния Левенштейна с кэшем в форме матрицы]
int levRecWithCache(char *s1, int i, char *s2, int j, int *cache, int cacheSize)
{
    if (i == 0)
        return j;
    else if (j == 0)
        return i;

    if (cache[i * cacheSize + j] < MAX_LIM)
        return cache[i * cacheSize + j];

    int flag;
    if (s1[i - 1] == s2[j - 1])
        flag = 0;
    else
        flag = 1;
    cache[i * cacheSize + j] = min(levRecWithCache(s1, i - 1, s2, j,\
     cache, cacheSize) + 1,\
                     levRecWithCache(s1, i, s2, j - 1, cache,\
                      cacheSize) + 1,\
                     levRecWithCache(s1, i - 1, s2, j - 1, cache,\
                      cacheSize) + flag);
    return cache[i * cacheSize + j];
}
\end{lstlisting}

\begin{lstlisting} [label=damLevRecWithoutCache, caption=Рекурсивный алгоритм поиска расстояния Дамерау-Левенштейна без кэша]
int damLevRecWithoutCache(char *s1, int i, char *s2, int j)
{
    if (i == 0)
        return j;
    else if (j == 0)
        return i;

    int flag;
    if (s1[i - 1] == s2[j - 1])
        flag = 0;
    else
        flag = 1;
    int res1 = damLevRecWithoutCache(s1, i - 1, s2, j) + 1;
    int res2 = damLevRecWithoutCache(s1, i, s2, j - 1) + 1;
    int res3 = damLevRecWithoutCache(s1, i - 1, s2, j - 1) + flag;

    int result = min(res1, res2, res3);
    if (i > 1 && j > 1 && s1[i - 1] == s2[j - 2] && s1[i - 2] == s2[j - 1])
        result = min(result, damLevRecWithoutCache(s1, i - 2, s2,\
         j - 2) + 1, MAX_LIM);
    return result;
}
\end{lstlisting}

\section{Тестирование ПО}

В таблице 3.1 приведены тестовые случаи для алгоритмов поиска редакционного расстояния.

\begin{table}[H]
	\begin{center}
		\caption{Тестовые случаи}
		\begin{tabular}{c|c|c|c|c}
			№ & Строка 1 & Строка 2 & Левенштейн & Д.-Левенштейн\\
			\hline
			1 & скат & кот & 2 & 2\\
			2 & "" & вуз & 3 & 3\\
			3 & дрова & двор & 3 & 3\\
			4 & музыка & мзуыка & 2 & 1\\
			5 & brown & born & 2 & 2\\
			6 & abcdef & badcef & 3 & 2\\
		\end{tabular}
	\end{center}
\end{table}

\section{Вывод}

На основе схем из конструкторского раздела были написаны реализации алгоритмов, а также было выполнено их тестирование.