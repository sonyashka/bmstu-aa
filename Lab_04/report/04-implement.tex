\chapter{Технологический раздел}

В данном разделе будут приведены требования к ПО, средства его реализации и листинга кода алгоритмов, а также рассмотрены тестовые случаи.

\section{Требования к ПО}

Программное обеспечение должно удовлетворять следующим требованиям:
\begin{itemize}
	\item ПО принимает вещественные пределы интегрирования и указатель на интегрируемую функцию;
	\item ПО возвращает вещественное число - результат интегрирования.
\end{itemize}

\section{Средства реализации} 

Для реализации ПО был выбран компилируемый язык C. В качестве среды разработки - QtCreator. Оба средства были выбраны из тех соображений, что навыки работы с ними были получены в более ранних курсах.

\section{Листинги кода}

Листинги 1 - 3 демонстрируют реализацию алгоритмов.

\captionsetup{singlelinecheck = false, justification=raggedright}
\begin{lstlisting}[label=refineIntegral, caption=Алгоритм последовательного вычисления значения определенного интеграла]
double refineIntegral(double a, double b, integralFunc func)
{
    int n = 2;
    double refRes = calculateIntegral(a, b, n, func);
    double res = 0;
    while (fabs(res - refRes) > EPS)
    {
        res = refRes;
        n *= 2;
        refRes = calculateIntegral(a, b, n, func);
    }
    return res;
}
\end{lstlisting}

\begin{lstlisting}[label=calculateIntegral, caption=Алгоритм вычисления значения определенного интеграла]
double calculateIntegral(double a, double b, int n, integralFunc func)
{
    double result = 0;
    double h = (b - a) / n;
    double curX = a;
    for (int i = 1; i < n; i++)
    {
        curX += h;
        result += func(curX);
    }
    result += (func(a) + func(b)) / 2;
    result *= h;
    return result;
}
\end{lstlisting}

\begin{lstlisting}[label=integralWithThreads, caption=Алгоритм параллельного вычисления значения определенного интеграла]
double integralWithThreads(int tNum, double a, double b, integralFunc func)
{
    double h = (b - a) / tNum;
    pthread_t threads[tNum];
    struct threadData threadData[tNum];
    for (int i = 0; i < tNum; i++)
    {
        threadData[i].ind = i;
        threadData[i].a = a;
        threadData[i].h = h;
        threadData[i].func = func;
        threadData[i].res = 0;
        pthread_create(&(threads[i]), NULL,\
                       oneThreadIntegral, &threadData[i]);
    }

    for (int i = 0; i < tNum; i++)
        pthread_join(threads[i], NULL);

    double res = 0;
    for (int i = 0; i < tNum; i++)
        res += threadData[i].res;
    return res;
}
\end{lstlisting}
\captionsetup{singlelinecheck = false, justification=centering}

\section{Тестирование ПО}

В таблице 3.1 приведены тестовые случаи для алгоритмов интегрирования, причем функция $f = x^2 + 4x - 21$.
Случай 1 - совпадение левого и правого пределов, случай 2 - левый предел больше правого, случаи 3-4 - значение интеграла ненулевое.

\begin{table}[H]
	\begin{center}
		\caption{Тестовые случаи}
		\begin{tabular}{l|l|l}
			№ & Пределы интегрирования & Результат\\
			\hline
			1 & 0 0 & 0\\ 
			\hline
			2 & -1 -8 & \text{Некорректные пределы интегрирования}\\
			\hline
			3 & 10 100 & 350910.000\\
			\hline
			4 & -1 3 & -58.666\\
		\end{tabular}
	\end{center}
\end{table}

\section{Вывод}

На основе схем из конструкторского раздела были написаны реализации требуемых алгоритмов, а также проведено их тестирвание.