\setcounter{page}{4}
\addchap{Введение}

Конвейер - механизм организации труда, когда производство изделия разбивается на стадии и конкретные работники закрепляются за своими стадиями (а часто и за линиями конвейера). Данный способ работы оказался эффективнее ранее используемых, в связи с чем такой термин как "конвейерная обработка" перекочевал и в программирование.

В его терминах конвейерная обработка данных - создание для определенных этапов решения задачи потоков, выполняющих свою работу на данных, полученных от предыдущей линии конвейера (другого потока). Использование многопоточности позволило существенно сократить время выполнения поставленных задач. 

Целью данной лабораторной работы является получение навыков конвейерной обработки данных. Для достижения поставленной цели необходимо выполнить следующие задачи:
\begin{itemize}
	\item изучить основы конвейерной обработки данных;
	\item составить схемы алгоритмов основной работы конвейера и его этапов;
	\item реализовать разработанные алгоритмы;
	\item провести сравнительный анализ конвейерной и последовательной реализаций по затрачиваемым ресурсам (время и память);
	\item описать и обосновать полученные результаты.
\end{itemize}