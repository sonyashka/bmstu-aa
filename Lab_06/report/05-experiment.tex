\chapter{Исследовательский раздел}

В данном разделе будет проведен сравнительный анализ алгоритмов по
времени и затрачиваемой памяти.

\section{Технические характеристики}

Тестирование выполнялось на устройстве со следующими характеристиками: 

\begin{itemize}
	\item Операционная система Ubuntu 20.04.3 LTS \cite{ubuntu}
	\item Память 16 Гб
	\item Процессор Intel Core i5-1135G7 11th Gen, 2.40 Гц \cite{intel}
\end{itemize}

Во время проведения эксперимента устройство не было нагружено сторонними задачами, а также было подключено к блоку питания.

Процессорное время работы алгоритмов было замерено с помощью функции process\_time
библиотеки time . Эта функция возвращает значение в долях секунды суммы системного и
пользовательского процессорного времени текущего процесса, не включая время, прошедшее
во время сна. Контрольная точка возвращаемого значения не определена, поэтому допустима
только разница между результатами последовательных вызовов. \cite{time}

\section{Оценка времени работы алгоритмов}

Замеры для каждого размера проводились 10 раз. В качестве результата взято среднее
время работы алгоритма при данном размере матриц. Матрицы заполнялись случайными
целыми числами в диапазоне от 0 до 10 000. Муравьиный алгоритм выполнялся при следующих параметрах: $\alpha$ = 0.5, p = 0.2, t\_iters = 8000, elits = count / 3.
Результаты замеров приведены в таблице 4.1.

\begin{table} [H]
	\caption{Таблица времени выполнения реализаций}
	\begin{center}
		\begin{tabular}{|c c c|} 
			\hline
			Кол-во вершин графа & Время полн. перебора(с) & Время муравьиного алг.(с)\\  
			\hline
			3 & 1.7 & 4.3e-05\\
			\hline
			5 & 5.0  & 0.00062\\
			\hline
			7 & 1.1e+01 & 0.036 \\
			\hline
			9 & 2.2e+01 & 3.1  \\
			\hline
			11 & 37 & 440\\
			\hline
		\end{tabular}
	\end{center}
	\label{tab:time}
\end{table}

Можно сделать вывод, что для графов, размером до 10 вершин алгоритм полного перебора оказывается выгоднее муравьиного алгоритма, однако при n = 11, второй алгоритм
работает быстрее первого в 11.9 раз. С увеличением количества вершин в графе это преимущество будет возрастать, так как время работы алгоритма полного перебора будет увеличиваться быстрее за счёт сложности алгоритма O(n!).

\section{Параметризация муравьиного алгоритма}

Для каждого набора $\alpha$, $\beta$, p найдём минимальное количество итераций t\_iter , для которого результат будет наиболее точным. Если за t\_iter = 10000 добиться правильного результата (заранее вычисленного с помощью алгоритма полного перебора) не удаётся, рассчитаем, на сколько полученный результат отличается от верного.

Подобный эксперимент проводится для трёх классов задач:
\begin{enumerate}
	\item многие вершины в графе не связаны между собой;
	\item расстояния в матрице расстояний незначительно отличаются друг от друга;
	\item расстояния в матрице значительно отличаются друг от друга.
\end{enumerate}
Полученные результаты представлены на таблицах 4.2 - 4.4.

\begin{table}[H]
\caption{Таблица параметризации для первого класса задач}
\centering
\subfloat{%
\begin{tabular}{|c c c c c|} 
			\hline
			$\alpha$ & $\beta$ & $p$ & $t_{iter}$ & $dif$\\  
			\hline
			0.0& 1.0& 0& 11500& 317.0\\
			\hline
			0.0& 1.0& 0.3& 8500& 0\\
			\hline
			0.0& 1.0& 0.5& 9000& 0\\
			\hline
			0.0& 1.0& 0.7& 11500& 317.0\\
			\hline
			0.0& 1.0& 1& 6500& 0\\
			\hline
			0.1& 0.9& 0& 7000& 0\\
			\hline
			0.1& 0.9& 0.3& 9000& 0\\
			\hline
			0.1& 0.9& 0.5& 4500& 0\\
			\hline
			0.1& 0.9& 0.7& 6000& 0\\
			\hline
			0.1& 0.9& 1& 6500& 0\\
			\hline
			0.2& 0.8& 0& 5000& 0\\
			\hline
			0.2& 0.8& 0.3& 5500& 0\\
			\hline
			0.2& 0.8& 0.5& 6000& 0\\
			\hline
			0.2& 0.8& 0.7& 6000& 0\\
			\hline
			0.2& 0.8& 1& 6000& 0\\
			\hline
			0.3&  0.7& 0& 6000& 0\\
			\hline
			0.3&  0.7& 0.3& 6500& 0\\
			\hline
			0.3&  0.7& 0.5& 4500& 0\\
			\hline
			0.3&  0.7& 0.7& 4500& 0\\
			\hline
			0.3& 0.7& 1& 5000& 0\\
			\hline
			0.4& 0.6& 0& 5500& 0\\
			\hline
			0.4& 0.6& 0.3& 6000& 0\\
			\hline
			0.4& 0.6& 0.5& 5000& 0\\
			\hline
			0.4& 0.6& 0.7& 4000& 0\\
			\hline
			0.4& 0.6& 1& 5000& 0\\
			\hline
			0.5& 0.5& 0& 2000& 0\\
			\hline
			0.5& 0.5& 0.3& 5500& 0\\
			\hline
			0.5& 0.5& 0.5& 3000& 0\\
			\hline
			0.5& 0.5& 0.7& 3000& 0\\
			\hline
			0.5& 0.5& 1& 2000& 0\\
			\hline
		\end{tabular}}%
\qquad\qquad% --- set horizontal distance between tables here
\subfloat{%
\begin{tabular}{|c c c c c|} 
			\hline
			$\alpha$ & $\beta$ & $p$ & $t_{iter}$ & $dif$\\  
			\hline
			0.6&  0.4&  0& 1500& 0\\
			\hline
			0.6&  0.4&  0.3& 5000& 0\\
			\hline
			0.6&  0.4&  0.5& 500& 0\\
			\hline
			0.6&  0.4&  0.7& 2500& 0\\
			\hline
			0.6&  0.4&  1& 500& 0\\
			\hline
			0.7&  0.3&  0& 1500& 0\\
			\hline
			0.7&  0.3&  0.3& 2500& 0\\
			\hline
			0.7&  0.3&  0.5& 5000& 0\\
			\hline
			0.7&  0.3&  0.7& 1500& 0\\
			\hline
			0.7&  0.3&  1& 1500& 0\\
			\hline
			0.8& 0.2&  0& 500& 0\\
			\hline
			0.8& 0.2&  0.3& 3500& 0\\
			\hline
			0.8& 0.2&  0.5& 500& 0\\
			\hline
			0.8& 0.2& 0.7& 500& 0\\
			\hline
			0.8& 0.2& 1& 1000& 0\\
			\hline
			0.9& 0.1& 0& 1000& 0\\
			\hline
			0.9& 0.1& 0.3& 1500& 0\\
			\hline
			0.9& 0.1& 0.5& 1000& 0\\
			\hline
			0.9& 0.1& 0.7& 1000& 0\\
			\hline
			0.9& 0.1& 1& 1000& 0\\
			\hline
			1.0& 0.0& 0& 500& 0\\
			\hline
			1.0& 0.0& 0.3& 1000& 0\\
			\hline
			1.0& 0.0& 0.5& 1000& 0\\
			\hline
			1.0& 0.0& 0.7& 1000& 0\\
			\hline
			1.0& 0.0& 1& 500& 0\\
			\hline
		\end{tabular}}
\end{table}

\begin{table}[H]
	\caption{Таблица параметризации для второго класса задач}
	\centering
	\subfloat{%
		\begin{tabular}{|c c c c c|} 
			\hline
			$\alpha$ & $\beta$ & $p$ & $t_{iter}$ & $dif$\\  
			\hline
			0.0& 1.0& 0& 8000& 0 \\
			\hline
			0.0& 1.0& 0.3& 6000& 0 \\
			\hline
			0.0& 1.0& 0.5& 5000& 0 \\
			\hline
			0.0& 1.0& 0.7& 7000& 0 \\
			\hline
			0.0& 1.0& 1& 5500& 0 \\
			\hline
			0.1& 0.9& 0& 6000& 0 \\
			\hline
			0.1& 0.9& 0.3& 6000& 0 \\
			\hline
			0.1& 0.9& 0.5& 7000& 0 \\
			\hline
			0.1& 0.9& 0.7& 5000& 0 \\
			\hline
			0.1& 0.9& 1& 6000& 0 \\
			\hline
			0.2& 0.8& 0& 6000& 0 \\
			\hline
			0.2& 0.8& 0.3& 6000& 0 \\
			\hline
			0.2& 0.8& 0.5& 5500& 0 \\
			\hline
			0.2& 0.8& 0.7& 5000& 0 \\
			\hline
			0.2& 0.8& 1& 6000& 0 \\
			\hline
			0.3 & 0.7& 0& 6000& 0 \\
			\hline
			0.3 & 0.7& 0.3& 5500& 0 \\
			\hline
			0.3 & 0.7& 0.5& 5000& 0 \\
			\hline
			0.3 & 0.7& 0.7& 6000& 0 \\
			\hline
			0.3 & 0.7& 1& 3000& 0 \\
			\hline
			0.4& 0.6& 0& 6000& 0 \\
			\hline
			0.4& 0.6& 0.3& 3500& 0 \\
			\hline
			0.4& 0.6& 0.5& 5000& 0 \\
			\hline
			0.4& 0.6& 0.7& 5500& 0 \\
			\hline
			0.4& 0.6& 1& 4000& 0 \\
			\hline
			0.5& 0.5& 0& 4000& 0 \\
			\hline
			0.5& 0.5& 0.3& 6000& 0 \\
			\hline
			0.5& 0.5& 0.5& 3500& 0 \\
			\hline
			0.5& 0.5& 0.7& 3500& 0 \\
			\hline
			0.5& 0.5& 1& 6000& 0 \\		
			\hline
		\end{tabular}}%
\qquad\qquad% --- set horizontal distance between tables here
\subfloat{%
\begin{tabular}{|c c c c c|} 
			\hline
			$\alpha$ & $\beta$ & $p$ & $t_{iter}$ & $dif$\\  
			\hline
			0.6 & 0.4& 0 & 1000& 0 \\
			\hline
			0.6  & 0.4 & 0.3& 5000& 0 \\
			\hline
			0.6 & 0.4 & 0.5& 3500& 0 \\
			\hline
			0.6 & 0.4 & 0.7& 2500& 0 \\
			\hline
			0.6 & 0.4 & 1& 1000& 0 \\
			0.7 & 0.3 & 0 & 1000 & 0 \\
			\hline
			0.7 & 0.3 & 0.3& 4500& 0 \\
			\hline
			0.7 & 0.3 & 0.5& 2000& 0 \\
			\hline
			0.7 & 0.3 & 0.7& 1500& 0 \\
			\hline
			0.7 & 0.3 & 1& 500& 0 \\
			\hline
			0.8& 0.2 & 0& 500& 0 \\
			\hline
			0.8& 0.2 & 0.3& 3000& 0 \\
			\hline
			0.8& 0.2 & 0.5& 1500& 0 \\
			\hline
			0.8& 0.2 & 0.7& 1000& 0 \\
			\hline
			0.8& 0.2 & 1& 1000& 0 \\
			\hline
			0.9& 0.1 & 0& 1000& 0 \\
			\hline
			0.9& 0.1 & 0.3& 2000& 0 \\
			\hline
			0.9& 0.1 & 0.5& 1000& 0 \\
			\hline
			0.9& 0.1 & 0.7& 500& 0 \\
			\hline
			0.9& 0.1 & 1& 500& 0 \\
			\hline
			1.0& 0.0& 0& 500& 0 \\
			\hline
			1.0& 0.0& 0.3& 4000& 0 \\
			\hline
			1.0& 0.0& 0.5& 500& 0 \\
			\hline
			1.0& 0.0& 0.7& 500& 0 \\
			\hline
			1.0& 0.0& 1& 500& 0 \\
			\hline
\end{tabular}}
\end{table}


\begin{table}[H]
	\caption{Таблица параметризации для третьего класса задач}
	\centering
	\subfloat{%
		\begin{tabular}{|c c c c c|} 
			\hline
			$\alpha$ & $\beta$ & $p$ & $t_{iter}$ & $dif$\\  
			\hline
			0.0& 1.0& 0& 9000& 0\\
			\hline
			0.0& 1.0& 0.3& 10500& 66875.0\\
			\hline
			0.0& 1.0& 0.5& 9500& 0\\
			\hline
			0.0& 1.0& 0.7& 10500& 66875.0\\
			\hline
			0.0& 1.0& 1& 9000& 0\\
			\hline
			0.1& 0.9& 0& 8000& 0\\
			\hline
			0.1& 0.9& 0.3& 5500& 0\\
			\hline
			0.1& 0.9& 0.5& 5500& 0\\
			\hline
			0.1& 0.9& 0.7& 5500& 0\\
			\hline
			0.1& 0.9& 1& 6500& 0\\
			\hline
			0.2& 0.8& 0& 4500& 0\\
			\hline
			0.2& 0.8& 0.3& 6000& 0\\
			\hline
			0.2& 0.8& 0.5& 7500& 0\\
			\hline
			0.2& 0.8& 0.7& 6000& 0\\
			\hline
			0.2& 0.8& 1& 5000& 0\\
			\hline
			0.3&  0.7& 0& 6000& 0\\
			\hline
			0.3&  0.7& 0.3& 7000& 0\\
			\hline
			0.3&  0.7& 0.5& 4000& 0\\
			\hline
			0.3&  0.7& 0.7& 6500& 0\\
			\hline
			0.3&  0.7& 1& 5500& 0\\
			\hline
			0.4& 0.6& 0& 5500& 0\\
			\hline
			0.4& 0.6& 0.3& 5000& 0\\
			\hline
			0.4& 0.6& 0.5& 6500& 0\\
			\hline
			0.4& 0.6& 0.7& 5500& 0\\
			\hline
			0.4& 0.6& 1& 5000& 0\\
			\hline
			0.5& 0.5& 0& 2000& 0\\
			\hline
			0.5& 0.5& 0.3& 4500& 0\\
			\hline
			0.5& 0.5& 0.5& 2500& 0\\
			\hline
			0.5& 0.5& 0.7& 5000& 0\\
			\hline
			0.5& 0.5& 1& 2000& 0\\
			\hline
	\end{tabular}}%
	\qquad\qquad% --- set horizontal distance between tables here
	\subfloat{%
		\begin{tabular}{|c c c c c|} 
			\hline
			$\alpha$ & $\beta$ & $p$ & $t_{iter}$ & $dif$\\  
			\hline
			0.6&  0.4&  0& 4000& 0\\
			\hline
			0.6&  0.4&  0.3& 3000& 0\\
			\hline
			0.6&  0.4&  0.5& 2500& 0\\
			\hline
			0.6&  0.4&  0.7& 3500& 0\\
			\hline
			0.6&  0.4&  1& 2500& 0\\
			\hline
			0.7&  0.3&  0& 1500& 0\\
			\hline
			0.7&  0.3&  0.3& 4000& 0\\
			\hline
			0.7&  0.3&  0.5& 2000& 0\\
			\hline
			0.7&  0.3&  0.7& 4500& 0\\
			\hline
			0.7&  0.3&  1& 500& 0\\
			\hline
			0.8& 0.2&  0& 1000& 0\\
			\hline
			0.8& 0.2&  0.3& 6000& 0\\
			\hline
			0.8& 0.2&  0.5& 2500& 0\\
			\hline
			0.8& 0.2&  0.7& 1500& 0\\
			\hline
			0.8& 0.2&  1& 1500& 0\\
			\hline
			0.9& 0.1&  0& 500& 0\\
			\hline
			0.9& 0.1&  0.3& 2500& 0\\
			\hline
			0.9& 0.1&  0.5& 3000& 0\\
			\hline
			0.9& 0.1&  0.7& 2000& 0\\
			\hline
			0.9& 0.1&  1& 500& 0\\
			\hline
			1.0& 0.0& 0& 1000& 0\\
			\hline
			1.0& 0.0& 0.3& 3500& 0\\
			\hline
			1.0& 0.0& 0.5& 3000& 0\\
			\hline
			1.0& 0.0& 0.7& 3000& 0\\
			\hline
			1.0& 0.0& 1& 2500& 0\\
			\hline
	\end{tabular}}
\end{table}

\section{Вывод}

Для матрицы расстояний, незначительно отличающихся друг от друга, можно сделать
следующие выводы:
\begin{enumerate}
	\item максимальное число итераций, при котором достигается верный результат, равен 8000 для $\alpha$ = 0, $\beta$ = 1, p = 0;
	\item наилучшие результаты достигаются при $\alpha$ >= 0.9, p > 0. При таких значениях параметров верный результат достигается за 500 итераций;
	\item в среднем значения количества итераций лучше для $\alpha$ >= 0.6.
\end{enumerate}

Для матрицы расстояний, значительно отличающихся друг от друга, можно сделать
следующие выводы:
\begin{enumerate}
	\item максимальное число итераций, при котором достигается верный результат, равен 9500 для $\alpha$ = 0, $\beta$ = 1, p = 0.5;
	\item для $\alpha$ = 0, $\beta$ = 1, p = 0.3 и $\alpha$ = 0, $\beta$ = 1, p = 0.7 за максимальное количество итераций 10000 не удалось достигнуть верного результата и полученное значение значительно отличается от него;
	\item наилучшие результаты достигаются при $\alpha$ = 0.7, p = 1. При таких значениях параметров верный результат достигается за 500 итераций;
	\item лля каждого набора $\alpha$, $\beta$ наилучший результат будет достигаться при наибольшем p. Это показывает, что для матриц расстояний, в которых расстояния сильно отличаются друг от друга, коэффициент испарения должен быть максимальным, чтобы усилить влияние длины пути на выбор пути.
\end{enumerate}