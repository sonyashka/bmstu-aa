\chapter{Технологический раздел}\label{sec:impl}

Раздел содержит листинги реализованных алгоритмов, требование к ПО
и тестирование реализованного программного обеспечения.

\section{Средства реализации} 
Для реализации ПО был выбран компилируемый многопоточный язык программирования Golang\cite{go}, поскольку язык отличается автоматическим управлением памяти. В качестве среды разработки была выбрана среда VS Code.

\section{Листинги кода}

На листинге \ref{lst:types} представлены используемые типы данных для реализации алгоритмов поиска

\captionsetup{singlelinecheck = false, justification=raggedright}
\begin{lstlisting}[label=lst:types,caption=Структуры данных]
type Dict_t map[int]string
type Segment_t Dict_t
\end{lstlisting}
Листинг \ref{lst:brute} демонстрирует реализацию алгоритма полного перебора.
\captionsetup{singlelinecheck = false, justification=raggedright}
\begin{figure}[H]
	\begin{lstlisting}[label=lst:brute,caption=Полный перебор]
	func BruteForce(dict Dict_t, key int) string {
	for i := 0; i < len(dict); i++ {
	if key - 1 == i {
	return dict[i]
	}
	} 
	return "ERROR:NOT_FOUND"
	}
	\end{lstlisting}
\end{figure}

На листинге \ref{lst:bin-wrap} представлена функция бинарного поиска:

\begin{figure}[H]
	\begin{lstlisting}[label=lst:bin-wrap,caption=Бинарный поиск]
	func BinarySearch(dict Dict_t, key int) string {
	var (left int
	right int
	mid int)
	
	left = 0
	right = len(dict) - 1
	
	if right != 0 {
	for left <= right {
	mid = left + (right - left) / 2
	if mid == key - 1 {
	return dict[mid]
	}
	if mid < key - 1 {
	left = mid + 1
	} else {
	right = mid - 1
	}
	}
	}
	return "ERROR:NOT_FOUND"
	}
	\end{lstlisting}
\end{figure}

На листинге \ref{lst:seg-bin-help} представлены вспомогательные функции для реализации алгоритма частотного анализа. В то время как на листинге \ref{lst:seg-bin} представлена сама реализация алгоритма частотного анализа.

\begin{figure}[H]
	\begin{lstlisting}[label=lst:seg-bin-help,caption= Вспомогательные функции для реализация алгоритма частотного анализа]
	var alphabet = string("abcdefghijklmnopqrstuvwxyz")
	var mask = make([]int, len(alphabet))
	
	func sort_keys(keys []int) {
	for i := 0; i < len(keys)-1; i++ {
	for j := 0; j < len(keys)-i-1; j++ {
	if keys[j] > keys[j+1] {
	swap_in_int_arr(keys, j, j+1)
	}
	}
	}
	}
	
	func init_mask() {
	for i := 0; i < len(alphabet); i++ {
	mask[i] = i
	}
	}
	
	func create_segments(size int) []Segment_t {
	segments := make([]Segment_t, size)
	for i := 0; i < size; i++ {
	segments[i] = make(Segment_t)
	}
	return segments
	}
	\end{lstlisting}
\end{figure}

\begin{figure}[H]
	\begin{lstlisting}[label=lst:seg-bin,caption=Реализация алгоритма частотного анализа]
	func divide_by_first_letter(segments []Segment_t, dict Dict_t) {
	for i := 0; i < len(dict); i++ {
	index := find_pos(strings.ToLower(string(dict[i][0])))
	segments[index][i] = dict[i]
	}
	}
	
	func sort_by_size(segments []Segment_t) {
	for i := 0; i < len(segments)-1; i++ {
	for j := 0; j < len(segments)-i-1; j++ {
	if len(segments[j]) < len(segments[j+1]) {
	swap_segs(segments, j, j+1)
	swap_in_int_arr(mask, j, j+1)
	}
	}
	}
	}
	
	func binary_search(segments []Segment_t, key int) string {
	var (left  int
	right int
	mid   int)
	
	for i := 0; i < len(segments); i++ {
	var keys = make([]int, 0)
	for key := range segments[i] {
	keys = append(keys, key)
	}
	sort_keys(keys)
	
	left = 0
	right = len(segments[i]) - 1
	
	if right != 0 {
	for left <= right {
	mid = left + (right-left)/2
	if keys[mid] == key {
	return segments[i][keys[mid]]
	}
	if keys[mid] < key {
	left = mid + 1
	} else {
	right = mid - 1
	}
	}
	}
	}
	return "ERROR:NOT_FOUND"
	}
	
	func FreqFind(dict Dict_t, key int) string {
	init_mask()
	segments := create_segments(len(alphabet))
	divide_by_first_letter(segments, dict)
	sort_by_size(segments)
	return binary_search(segments, key-1)
	}
	\end{lstlisting}
\end{figure}

\section{Тестирование ПО}
Результаты тестирования ПО приведены в таблице \ref{table:testing}, в которой используются следующие обозначения:
\begin{enumerate}
	\item ПП - алгоритм полного перебора;
	\item БП - алгоритм бинарного поиска;
	\item ЧА - алгоритм частотного анализа;
\end{enumerate} 

\begin{table}[H]
	\captionsetup{singlelinecheck = false, justification=raggedright}
	\caption{Тестирование ПО}
	\renewcommand{\arraystretch}{2}
	\begin{center}
		\begin{tabular}{|c|c|c|c|c|c|}
			\hline
			Исходный словарь &  Ключ & ПП & БП & ЧА & Ожидание \\ \hline
			1: 'I', 2:'want', 3:'sleep' & 1 & 'I' & 'I' &'I'& 'I'\\
			1: 'I', 2:'want', 3:'sleep' & 2 & 'want' & 'want' &'want'& 'want'\\
			1: 'I', 2:'want', 3:'sleep', 4:'sleep'& 3 & 'sleep' & 'sleep' &'sleep'& 'sleep'\\
			1: 'I', 2:'want', 3:'sleep', 2:'want'& 3 & 'want' & 'want' &'want'& 'want'\\
			\hline
		\end{tabular}
	\end{center}
	\label{table:testing}
\end{table}

\section{Вывод}
Было написано и протестировано программное обеспечение для решения поставленной задачи.