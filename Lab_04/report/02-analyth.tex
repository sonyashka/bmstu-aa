\chapter{Аналитический раздел}

В данном разделе будут представлены описания последовательного и параллельного вариантов алгоритма вычисления определенного интеграла методом трапеций.

\section{Вычисление определенного интеграла методом трапеций}

Определенный интеграл - одно из основных математических понятий, применимое при вычислении площади под графиком заданной функции. Для его вычисления необходимо разбить отрезок, по которому будет интегрироваться функция, на множество маленьких отрезков и просуммировать значения функции. Это общий алгоритм вычисления определенного интеграла.

\begin{equation}
	I = \int\limits_a^b f(x)dx 
\end{equation}

Чтобы повысить точность вычислений, можно использовать метод трапеций \cite{int}. Его суть заключается в том, что на всех интервалах кроме крайних левого и правого вычисляется значение функции, суммируется, после прибавляется усредненное значение неучтенных интервалов и домножается на шаг интервалов.

Представим это в виде математической формулы. Если $a$ и $b$ - левая и правая границы интервалов, $n$ - количество интервалов, $h = (b - a) / n$ - шаг аргумента функции, $x_{i}$ - значение аргумента на шаге $i$, $f(x_{i})$ - значение функции на шаге $i$. Тогда определенный интеграл может быть вычислен по следующей формуле:

\begin{equation}
	I = h \cdot (\frac{f(a) + f(b)}{2} + \displaystyle\sum_{i = 1}^{n - 1} f(x_{i}))
\end{equation}

\section{Параллельное вычисление определенного интеграла}

Так как в алгоритме происходит разбиение исходного отрезка на несколько, и вычисления на них могут происходить независимо, целесообразно произвести распараллеливание по отрезкам интегрирования. Причем количество данных отрезков будем определять как количество создаваемых потоков. На каждом отрезке поток \cite{threads} будет вычислять значение определенного интеграла, и после того, как все потоки завершаться, будет произведено суммирование. 

\section{Вывод}

В данном разделе были рассмотренны принципы работы последовательного и параллельного алгоритмов вычисления определенного варианта. Полученных знаний достаточно для разработки выбранных алгоритмов. На вход алгоритмам будут подаваться начало и конец интервала интегрирования, а также количество интервалов разбиения и исходная фукция. Реализуемое ПО будет работать в пользовательском режиме (вывод отсортированного массива тремя методами), а также в экспериментальном (проведение замеров времени выполнения алгоритмов).