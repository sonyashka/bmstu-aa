\chapter{Аналитический раздел}

В данном разделе будут представлены опсиания алгоритмов выбранных сортировок: пузырьком, вставками и выбором.

\section{Сортировка пузырьком}

Сортировка пузырьком или простыми обменами \cite{bubbleSort} (англ. \textit{bubble sort}) - простой алгоритм сортировки, работающий по следующему приниципу: за каждый проход элементы последовательно сравниваются попарно и, если порядок в паре неверный, выполняется перестановка элементов. Проходы по массиву повторяются N-1 раз или до тех пор, пока на очередном проходе не окажется, что обмены больше не нужны, что означает — массив отсортирован. При каждом проходе алгоритма по внутреннему циклу, очередной наибольший элемент массива ставится на своё место в конце массива рядом с предыдущим «наибольшим элементом», а наименьший элемент перемещается на одну позицию к началу массива («всплывает» до нужной позиции, как пузырёк в воде — отсюда и название алгоритма).

\section{Сортировка вставками}

Сортировка вставками \cite{insertionSort} (англ. \textit{insertion sort}) - алгоритм сортировки, в котором элементы входной последовательности просматриваются по одному, и каждый новый поступивший элемент размещается в подходящее место среди ранее упорядоченных элементов.

На вход алгоритма подается массив A из n элементов. Сортируемые элементы также называются ключами. Каждый ключ последовательно сравнивается с уже отсортированной частью массива. Если для конкретного ключа выполняется следующее условие

\begin{equation}
	K_{i - 1} \leq K_{ind} \leq K_{i}
\end{equation}

где $K_{i - 1}$, $K_{i}$ - ключи уже отсортированной последовательности, а $K_{ind}$ - ключ, для которого осуществляется поиск позиции. Когда условие выполняется элементы больше текущего сдвигаются вправо на одну позицию, а текущий помещается на нужную позицию. Сортировка заканчивается когда неотсортированная часть становится пустой.

\section{Сортировка выбором}

Сортировка выбором \cite{selectionSort} (англ. \textit{selection sort}) - алгоритм сортировки, основанный на поиске минимального элемента на каждом шаге. Когда такой будет найден, он помещается в конец отсортированной части массива и в слудющих итерациях учитываться уже не будет. Изнчально отсортированная часть пуста. Сортировка заканчивается, когда все элементы рассмотренны. 

Данный алгоритм похож на сортировку пузырьком, за тем исключением, что в пузырьке все элементы сравниваются попарно и после соответствующих обменов элемент оказывается в нужной части массива; в этом же алгоритме сначала происходит поиск минимума, после чего перестановка элемента в нужное место. Это позволяет сократить количество обменов значениями. 

\section{Вывод}

В данном разделе были рассмотренны принципы работы трех алгоритмов сортировки. Полученных знаний достаточно для разработки выбранных алгоритмов. На вход алгоритмам будут подаваться массив элементов и его размер. Реализуемое ПО будет работать в пользовательском режиме (вывод отсортированного массива тремя методами), а также в экспериментальном (проведение замеров времени выполнения алгоритмов).