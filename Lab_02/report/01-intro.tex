\setcounter{page}{4}
\addchap{Введение}

В настоящее время компьютеры оперируют множеством типов данных. Одним из них являются матрицы. Для некоторых алгоритмов необходимо выполнять их перемножение. Данная операция является затратной по времени, имея сложность порядка $O(N^3)$. Поэтому Ш. Виноград создал свой алгоритм умножения матриц, который являлся ассимптотически самым быстрым из всех. Однако преимущества появляются только на матрицах большого размера, настолько что современная вычислительная техника не способна оперировать таким объемом данных.

Целью данной лабораторной работы является анализ алгоритмы перемножения матриц Винограда. Для достижения поставленной цели необходимо выполнить следующие задачи:
\begin{itemize}
	\item Изучить выбранные алгоритмы умножения матриц и способы оптимизации;
	\item Составить схемы рассмотренных алгоритмов;
	\item Реализовать разработанные алгоритмы умножения матриц;
	\item Провести сравнительный анализ алгоритмов по затрачиваемым ресурсам (время и память);
	\item Описать и обосновать полученные результаты.
\end{itemize}